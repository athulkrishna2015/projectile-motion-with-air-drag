\documentclass[12pt,a4paper]{article}
\usepackage[
% Some remarks:
% * drivers like 'pdftex' that can be detected automatically
%   are not necessary
% * breaklinks is rather an internal option.
%   If a driver does not support it, then forcing the option
%   let the text break across lines, but also the link
%   areas are "broken". If the driver supports the option,
%   then the option is enabled anyway.
% * Information entries should be set outside,
%   because LaTeX expands the package options,
%   hyperref does not like them, if they are
%   prematurely expanded.
% * Hyperref has a new option for hiding links: hidelinks
hidelinks,
pagebackref,
bookmarksopen,
bookmarksnumbered,
]{hyperref}
\hypersetup{
	pdfauthor={...},
	% ...
}
% Adding package bookmark improves bookmarks handling.
% More features and faster updated bookmarks.
\usepackage{bookmark}
\usepackage{amsmath}
\usepackage{graphicx}

\usepackage{listings}
\usepackage{xcolor}

\definecolor{mGreen}{rgb}{0,0.6,0}
\definecolor{mGray}{rgb}{0.5,0.5,0.5}
\definecolor{mPurple}{rgb}{0.58,0,0.82}
\definecolor{backgroundColour}{rgb}{0.95,0.95,0.92}

\lstdefinestyle{CStyle}{
    backgroundcolor=\color{backgroundColour},   
    commentstyle=\color{mGreen},
    keywordstyle=\color{magenta},
    numberstyle=\tiny\color{mGray},
    stringstyle=\color{mPurple},
    basicstyle=\footnotesize,
    breakatwhitespace=false,         
    breaklines=true,                 
    captionpos=b,                    
    keepspaces=true,                 
    numbers=left,                    
    numbersep=5pt,                  
    showspaces=false,                
    showstringspaces=false,
    showtabs=false,                  
    tabsize=2,
    language=C
}
\linespread{2}
%	\newpage% or \cleardoublepage
% \pdfbookmark[<level>]{<title>}{<dest>}
\pdfbookmark[section]{\contentsname}{toc}
\begin{document}
    \title{\textbf{C program to study Projectile motion with air drag using Euler method}\\Term Paper}
	\author{Athulkrishna S V \\20mscpy08}
	\date{\today}
	\begin{figure}
		\centering
		\includegraphics[width=.6\textwidth]{logo.png}
	\end{figure}
	\maketitle

	\newpage

	\begin{abstract}
		In this Term Paper we will discus how numerical method can be used to Predict the motion of a projectile with air drag Using Euler method
	\end{abstract}
	
	\tableofcontents
	
	\newpage

	\section{Introduction}
	Need of numerical analysis: Most of the problems in mechanics start with the questions of motion and analysis of trajectory of motion. 
	During the process of solving the problems with analytical methods, we are forced to neglect many parameters to avoid complexity.
	If external parameter controlling the motion are large numbers, The error is also large.
	A Solution incorporating all these parameters will be highly complicated\\
	In numerical analysis we can solve this problem by step by step, including all parameters, which we like to incorporate. By the fundamentals of differentiation,  
	\begin{align}
		v_t=\lim_{h \to 0} = \frac{x_{t+h}-x_t}{h}\\
		a_t=\lim_{h \to 0} = \frac{v_{t+h}-v_t}{h}
	\end{align}
	This is true only if \(h \rightarrow 0\) this means \(h\) must be infinitesimally small. This can be made reality only with a computer.
	so when \(h\) become very small, above set of equations become
	\begin{align}
		x_{t+h}&=x_t+hv_t\\
		v_{t+h}&=v_t+ha_t\\
		a&=\frac{F(x,v,t)}{m}
	\end{align}
	using this equations we can estimate the acceleration, velocity, displacement in step by step using a computer.
	\section{One Dimensional motion }
	\subsection{Position and velocity of a body - Euler method}
	we know from fundamentals,
	\begin{equation}
		\Delta v=\Delta t \times a
	\end{equation}
	if
	\begin{align}
		\Delta t&= h \\
		\Delta v &= ha\\
		v_2-v_1 &= ha_1\\
		v_2&=v-1+ha_1
	\end{align}
		similarly
	\begin{equation}
		v_3=v_2+ha_2
	\end{equation}
	Extending this to particular time element
	\begin{equation}
		v_{i+1}=v_i+ha_i
	\end{equation}
	From fundamentals
	\begin{align}
		v&=\frac{\Delta x}{\Delta t}=\frac{\Delta x}{h}\\
		\Delta x &= hv \\
		x_2-x_1 &= hv_1\\
		x_2&=x-1+hv_1
	\end{align}
	similarly
	\begin{equation}
		x_3=x-2+hv_1
	\end{equation}
	Extending this to particular time element
	\begin{equation}
		x_{i+1}=x_i+ha_i
	\end{equation}
	Divide the total time into small elements. From the initial value find the velocity and position during that element.
	Using that value, we can find the velocity and  position in the next time element. This can be continued till last time slot to get the final value.
	This is the Euler method.
	\subsection{Freely falling body with air drag}
	When a body moves downward, The air drag is proportional to square of the velocity.
	\begin{align}
		F_{air}&\propto v^2\\
		F_{air}&=kv^2\\
		F_{air}&=\frac{1}{2}c \pi \rho r^2 v^2
	\end{align}
	where\\
	\(c = \) Drag Coefficient\\
	\(\rho = \) Density of Medium\\
	\(r = \) Radius of the body\\
	Net downward force 
	\begin{align}
		F&=mg-\frac{1}{2}C \pi \rho r^2 v^2\\
		a&= g-Kv^2
	\end{align}
	where \(K=\frac{1}{2m}c\pi \rho r^2\)

	\section{Two Dimensional Motion }
	When we are making Two Dimensional analysis, the analysis methodology is same as one dimensional, but we need to find the acceleration, velocity and displacement in x direction and y direction Separately.

	from Euler method in One Dimension
	\begin{align}
		v_{i+1}&=v_i+ha_i\\
		x_{i+1}&=x_i+hv_i
	\end{align}
	Extending to x and y direction separately
	\begin{align}
		v_{x(i+1)}&=v_{xi}+ha_{xi}\\
		v_{y(i+1)}&=v_{yi}+ha_{yi}\\
		x_{x(i+1)}&=v_{xi}+hv_{xi}\\
		v_{y(i+1)}&=v_{yi}+hv_{yi}
	\end{align}
	A two dimensional analysis is equivalent to two one dimensional analysis. The analysis can be repeated with the correction due to variation in acceleration due to gravity,force of buoyancy,viscus force and air drag.

	\subsection{Projectile motion by Euler method }
	A projectile is a body projected with initial velocity at an angle with horizontal. In this case initial velocity and acceleration in x and y direction  are different. Time of flight is same inx anf y direction.
	consider the influence of gravity only. Then 
	\begin{align}
		F_x=&0		&F_y=&-mg\\
		F_y=&0		&F_y=&-g
	\end{align} 
	Since the projectile is projected at an angle, The initial velocity can be split into two components. They they are \(v_0 cos \theta \) along x axis and \(v_0 sin \theta \) along y axis. then as Discussed before. By Euler method,\\
	\subsubsection{Along x direction}
	Acceleration=0\\
	initial velocity = \(v_0 cos \theta\)\\
	\(v_{x(i+1)}=v_{xi}+ha_{xi}\)\\
	\(x_{x(i+1)}=v_{xi}+hv_{xi}\)\\
	\subsubsection{Along y direction}
	Acceleration=-g\\
	initial velocity = \(v_0 sin \theta\)\\
	\(v_{y(i+1)}=v_{yi}+ha_{yi}\)\\
	\(x_{y(i+1)}=v_{yi}+hv_{yi}\)\\


	Using this formula the position and velocity at any stage can be calculated. In the case of a projectile, the maximum value of displacement in x direction is called maximum range. The maximum value of displacement in y direction is called maximum hight.
	
	\subsection{Projectile motion with air drag}
	On practical situations, the air drag will oppose the projectile motion. So we getter better estimate of trajectory, we have to incorporate the variation of acceleration force due to air drag. As discussed earlier, air drag,
	\begin{equation}
		F_d=\frac{1}{2}C\pi \rho r^2 v^2 
	\end{equation}
	The effect of air drag can be split into two components \(F_d cos \phi \) along x axis anf \(F_d sin \phi \) along y axis where \(\phi \)is the angle by velocity component with x direction at any instant.
	\subsubsection{x-componet }
		Force due to Earths gravity =0\\
		Air drag = \(-F_d cos \phi = \frac{1}{2}C\pi \rho r^2 v^2 \cos \phi \) = \(-\frac{1}{2}C\pi \rho r^2 v^2 \)\\
		Net force at any instant = \(-\frac{1}{2}C\pi \rho r^2 v^2 \cos \phi \)\\
		Net acceleration at any instant = \(-\frac{C\pi \rho r^2 v^2 \cos \phi }{2}\)= \(-kv^2 cos \phi \) \\
		Where \(k=\frac{1}{2m}c\pi \rho r^2\)\\
		At any instant, \(cos \phi = \frac{v_x}{v}\)\\
		Substituting \\
		\(a_x=-kv^2\frac{v_x}{v}=-kvv_x=-kv\sqrt{v_x^2+v_y^2}\)
		\subsubsection{y-component}
		Force due to Earths gravity =\(-mg\)\\
		Air drag = \(-F_d sin \phi = \frac{1}{2}C\pi \rho r^2 v^2 \sin \phi \) = \(-\frac{1}{2}C\pi \rho r^2 v^2 \)\\
		Net force at any instant = \(-mg-\frac{1}{2}C\pi \rho r^2 v^2 \sin \phi \)\\
		Net acceleration at any instant = \(-g-\frac{C\pi \rho r^2 v^2 \sin \phi }{2}\)= \(-g-kv^2 cos \phi \) \\
		Where \(k=\frac{1}{2m}c\pi \rho r^2\)\\
		At any instant, \(sin \phi = \frac{v_y}{v}\)\\
		Substituting \\
		\(a_x=-g-kv^2\frac{v_y}{v}=-kvv_y=-kv\sqrt{v_x^2+v_y^2}\)
	From acceleration we can estimate the velocity and position along x and y axis at any ins by Euler method.
	\section{programme for Projectile motion}
	\subsection{C programme}
	\begin{lstlisting}[style=CStyle]
		#include <stdio.h>
		#include <math.h>
		int main()
		{
			float v0,ang,tf,h,drag,ro,rad,mass,k,ay,ax,vy,vx,x,y,t;
			printf("Enter initial velocity ");
			scanf("%f",&v0);
			printf("Enter angle of projection ");
			scanf("%f",&ang);
			printf("Enter time ");
			scanf("%f",&tf);
			printf("Enter the time step ");
			scanf("%f",&h);
			printf("Enter air drag ");
			scanf("%f",&drag);
			printf("Enter Dencity ");
			scanf("%f",&ro);
			printf("Enter radis of the body ");
			scanf("%f",&rad);
			printf("Enter the mass of the body ");
			scanf("%f",&mass);
			k=0.5*3.1415926*drag*ro*rad*rad/mass;
			// printf("%f",k);
			ang=ang*3.1415926/180;
			ax=0;
			ay=-9.8;
			vx=v0*cos(ang);
			vy=v0*sin(ang);
			x=y=t=0;
			printf("Time   X-acceleration   Y-acceleration  X-Velocity  Y-Velocity  X-Position  Y-position\n");
			printf("%6.3f  %6.3f  %6.3f  %6.3f  %6.3f  %6.3f  %6.3f \n",t,ax,ay,vx,vy,x,y);
			t=t+h;
			while (t<=tf)
			{
				ax=-k*vx*sqrt(vx*vx+vy*vy);
				ay=-9.8-k*vy*sqrt(vx*vx+vy*vy);
				vx=vx+ax*h;
				vy=vy+ay*h;
				x=x+vx*h;
				y=y+vy*h;
				printf("Time   X-acceleration   Y-acceleration  X-Velocity  Y-Velocity  X-Position  Y-position\n");
				printf("%6.3f  %6.3f  %6.3f  %6.3f  %6.3f  %6.3f  %6.3f \n",t,ax,ay,vx,vy,x,y);
				t=t+h;
				ax=ax;
				ay=ay;
			}
			printf("Thank You"); 
		}
		
	\end{lstlisting}
	\subsection{Example }
	A body is projected with a velocity of 10 m/s at an angle \(60^0\). Considering thw effect of air drag Tabulate the position and velocity for the first 0.6 Sec by Euler method with time step 0.1 sec. Coefficient of drag  = 0.3, Density of air \(1.2 kg m^{-1}\), Radius of the body = 0.5m , mass of the body = 1kg.
	\begin{lstlisting}[style=CStyle]
		Enter initial velocity 10
		Enter angle of projection 60
		Enter time .6
		Enter the time step .1
		Enter air drag .3 
		Enter Dencity 1.2
		Enter radis of the body .5
		Enter the mass of the body 1
		Time   X-acceleration   Y-acceleration  X-Velocity  Y-Velocity  X-Position  Y-position
		 0.000   0.000  -9.800   5.000   8.660   0.000   0.000 
		 0.100  -7.069  -22.043   4.293   6.456   0.429   0.646 
		 0.200  -4.706  -16.876   3.823   4.768   0.812   1.122 
		 0.300  -3.303  -13.920   3.492   3.376   1.161   1.460 
		 0.400  -2.398  -12.119   3.252   2.164   1.486   1.677 
		 0.500  -1.796  -10.995   3.073   1.065   1.793   1.783 
		 0.600  -1.413  -10.290   2.932   0.036   2.086   1.787 
	\end{lstlisting}

\section{Conclusion }
We were able to predict the motion of an Projectile with air drag with ease using numerical method. If we where to calculate this analytically it will be a hard process. Numerical methods using Programming make our life easier.
\section{Reference}
\begin{enumerate}
	\item \textbf{Introductory methods of numerical analysis} \textit{Fifth Edition}	 S.S. Sastry 
	\item \textbf{Let Us C } \textit{Fifth Edition} Yashavant P. Kanetkar
	\item \textbf{Numerical-methods} \textit{} E Balaguruswamy
	\item \href{https://github.com/athulkrishna2015/projectile-motion-with-air-drag}{Github}. 
\end{enumerate}
	\end{document}